

%\documentclass{wissdoc}
\documentclass[oneside]{wissdoc}
% ----------------------------------------------------------------
% Diplomarbeit - Hauptdokument
% ----------------------------------------------------------------
% wissdoc Optionen: draft, relaxed, pdf, oneside --> siehe wissdoc.cls
% ------------------------------------------------------------------
% Packages für Deckblatt

\usepackage[absolute]{textpos} 	%Textboxen an absolute Position setzen
\usepackage{setspace}						%Zeilenabstand anpassen
\usepackage{color}							%Farbige Schrift
\usepackage[dvipsnames,table,xcdraw]{xcolor}
\usepackage{graphicx}						%Einbinden von Grafiken


% bib stuff
\usepackage[autostyle]{csquotes}

\usepackage[
    backend=biber,
    style=ieee,
    sortlocale=de_DE,
    natbib=true,
    url=true,
    doi=true,
    eprint=false
]{biblatex}
\addbibresource{literatur.bib}

\usepackage[]{hyperref}
\hypersetup{
    colorlinks=false,
    linkbordercolor=gray,
    urlbordercolor=blue
    %urlcolor=blue,
    %linkcolor=black
}


% Weitere packages: (Dokumentation dazu durch "latex <package>.dtx")
% \usepackage{varioref}
% \usepackage{verbatim}
% \usepackage{float}    %z.B. \floatstyle{ruled}\restylefloat{figure}
\usepackage{subfigure}
\usepackage[ngerman]{babel}
\usepackage[T1]{fontenc}
\usepackage[utf8]{inputenc}
\usepackage{url}
\usepackage[acronym,toc]{glossaries}

\makeglossaries

\usepackage{xparse}
\DeclareDocumentCommand{\newdualentry}{ O{} O{} m m m m } {
  \newglossaryentry{gls-#3}{name={#4},text={#4\glsadd{#3}},
    description={#6},#1
  }
  \makeglossaries
  \newacronym[see={[Glossary:]{gls-#3}},#2]{#3}{#4}{#5\glsadd{gls-#3}}
}

\loadglsentries{glossaries}

%\usepackage[footnote]{acronym}

% Zeilenabstand nach Vorgabe - Falls gefordert
%\setstretch{1,3} 

% Inhaltsangabe auf Unterabschnitte(2 Ebenen) begrenzen
\setcounter{tocdepth}{2}


% \usepackage{color}    % Farbiger/grauer Text
% \usepackage{colortbl}   % Farbige/graue Tabellenzeilen und -spalten!! <--
% \usepackage{fancybox} % für schattierte,ovale Boxen etc.
% \usepackage{tabularx} % automatische Spaltenbreite
% \usepackage{supertab} % mehrseitige Tabellen
%% ---------------- end of usepackages -------------

%% Informationen für die PDF-Datei
\hypersetup{pdfauthor={Domenik Müller},%
            pdftitle={Bachelorarbeit: FPGA-Event-Recorder},%
            pdfsubject={Design und Implementierung eines FPGA-Event-Recorders mit der freien IceStorm-Toolchain},%
            pdfkeywords={FPGA, Open Source, IceStorm, iCE40, Lattice},%
            pdfproducer={LaTeX},%
            pdfcreator={pdfLaTeX}
}

% Macros, nicht unbedingt notwendig
\input{macros}

% Print URLs not in Typewriter Font
\def\UrlFont{\rm}

\newcommand{\blankpage}{% Leerseite ohne Seitennummer, nächste Seite rechts
 \clearpage{\pagestyle{empty}\cleardoublepage}
}

%% Einstellungen für das gesamte Dokument

% Trennhilfen
% Wichtig!
% Im german-paket sind zusätzlich folgende Trennhinweise enthalten:
% "- = zusätzliche Trennstelle
% "| = Vermeidung von Ligaturen und mögliche Trennung (bsp: Schaf"|fell)
% "~ = Bindestrich an dem keine Trennung erlaubt ist (bsp: bergauf und "~ab)
% "= = Bindestrich bei dem Worte vor und dahinter getrennt werden dürfen
% "" = Trennstelle ohne Erzeugung eines Trennstrichs (bsp: und/""oder)

% Trennhinweise fuer Woerter hier beschreiben
\hyphenation{
% Pro-to-koll-in-stan-zen
% Ma-na-ge-ment  Netz-werk-ele-men-ten
% Netz-werk Netz-werk-re-ser-vie-rung
% Netz-werk-adap-ter Fein-ju-stier-ung
% Da-ten-strom-spe-zi-fi-ka-tion Pa-ket-rumpf
% Kon-troll-in-stanz
}

%Tabellen Kommandos
\newcolumntype{L}[1]{>{\raggedright\arraybackslash}p{#1}}
\newcolumntype{C}[1]{>{\centering\arraybackslash}p{#1}}
\newcolumntype{R}[1]{>{\raggedleft\arraybackslash}p{#1}}

% Index-Datei öffn
\makeindex



\begin{document}

\pagenumbering{gobble}

%%% Deckblatt - Hochschule Augsburg
%%%Deckblatt

\textblockorigin{20mm}{30mm}

\thispagestyle{empty}\null
%%%%Logo - Hochschule Augsburg - Informatik
\begin{textblock}{10}(8.0,1.1)
\begin{figure}[h]
	\centering
		\includegraphics[width=0.45\textwidth]{logos/hsa_informatik_logo_lq.pdf}
\end{figure}

\end{textblock}

%%% Text unter Logo
\begin{textblock}{15}(12.43,2.1)
	\LARGE
	\textsf{
		\textbf{\textcolor[rgb]{1,0.41,0.13}{\\
			\begin{flushleft}
				Fakultät für\\
				Informatik\\
			\end{flushleft}
			}
		}
	}
\end{textblock}

%%%%Textbox links - Informationen
\begin{textblock}{15}(2,1.4)
	%\LARGE
	\begin{flushleft}
		\begin{spacing} {1.2}
			\LARGE	
				\vspace{140pt}
				\textcolor[rgb]{1,0.41,0.13}{\\
				\textbf{Bachelorarbeit}}\\
				\vspace{10pt}
			\LARGE   \textbf{Desgin und Implementierung \\eines FPGA-Event-Recorders mithilfe \\der freien IceStorm-Toolchain} \\			
			\Large
				\vspace{30pt}		
				Studienrichtung\\
				Technische Informatik\\
				\vspace{30pt}
				Domenik Müller\\
				\vspace{30pt}		
				\vspace{80pt}		
			\Large
				Prüfer: Prof. Dr. Hubert Högl\\
				Zweitprüfer: Prof. Dr. Alexander von Bodisco\\ 
				\vspace{7pt}		
				Abgabedatum: 20.06.2018\\
			\end{spacing}
		\end{flushleft}
		
\end{textblock}



%%%%Textbox rechts - Hochschule
\begin{textblock}{5}(12.45,9.0)
	\scriptsize
	\textcolor[rgb]{1,0,0}{\\
		\begin{flushleft}
			\begin{spacing} {1.3}
				Hochschule f\"ur angewandte\\
				Wissenschaften Augsburg\\
				\vspace{4pt}
				An der Hochschule 1\\
				D-86161 Augsburg\\
				\vspace{4pt}
				Telefon +49 821 55 86-0\\
				Fax +49 821 55 86-3222\\
				www.hs-augsburg.de\\
				info@hs-augsburg.de
			\end{spacing}
		\end{flushleft}
		}
\end{textblock}


%%%%Textbox rechts unten - Fakultät und Autor
\begin{textblock}{5}(12.45,11.5)
	\scriptsize
		\begin{flushleft}
			\begin{spacing} {1.3}
				Fakult\"at f\"ur Informatik\\
				Telefon +49 821 55 86-3450\\
				Fax \hspace{10pt} +49 821 55 86-3499\\
				\vspace{6pt}
				Verfasser der Diplomarbeit\\
				Domenik Müller\\
				Am Eser 3\\
				86150 Augsburg\\
				Telefon +49 821 44 92 57 54\\
				domenik.mueller@hs-augsburg.de\\
			\end{spacing}
		\end{flushleft}
	\end{textblock}
\pagebreak
  %<-- Nach Vorgabe der HS Augsburg


%
%% ++++++++++++++++++++++++++++++++++++++++++
%% Verzeichnisse
%% ++++++++++++++++++++++++++++++++++++++++++

\section*{Zusammenfassung}

\ldots

\section*{Abstract}

This work of magnificience \ldots

\ifnotdraft{
\tableofcontents
% Leere Seite bei zweiseitigem Druck
%\ifnotonesideelse{\blankpage}{}
%\listoffigures
%% Leere Seite bei zweiseitigem Druck
%\ifnotonesideelse{\blankpage}{}
%\listoftables
%% Leere Seite bei zweiseitigem Druck
%\ifnotonesideelse{\blankpage}{}
}

\clearpage


%% ++++++++++++++++++++++++++++++++++++++++++
%% Hauptteil
%% ++++++++++++++++++++++++++++++++++++++++++
\graphicspath{{figures/}}
\pagenumbering{arabic}


%%% Ab hier eigene Kapitel einfügen
%%% Kapitel sind analog zur Wordvorlage zu wählen
\chapter{Einführung}

\label{ch:Einfuehrung}

Mikrocontroller werden heute in Gebrauchsgegenständen aller Art verbaut und werden den Anforderungen entsprechend immer leistungsstärker und damit unter anderem auch schneller. Selbst einfache Mikrocontroller arbeiten oft mit einer Geschwindigkeit im mehrstelligen Megaherz Bereich (sprich: mehrere Millionen Takte pro Sekunde). Im Unterschied zu klassischen PC-Systemen werden an Mikrocontroller oft Echtzeit-Anforderungen gestellt, das heißt Ergebnisse müssen zuverlässig innerhalb einer vorbestimmten Zeitspanne geliefert werden\cite{wiki:echtzeit}. Dabei wird auch die Hardware von Mikrocontrollern zunehmend komplexer und es werden vermehrt Mehrprozessor-Systeme verwendet, die angepasste und mitunter unübersichtlichere Programmiertechniken erfordern.\\ 
Dementsprechend werden für die Entwicklung von Mikrocontroller-Systemen (aber auch von digitalen Systemen im allgemeinen) Werkzeuge benötigt, mit denen Signale mit hoher zeitlicher Auflösung erfasst und analysiert werden können.\\
Ergänzend zu Simulations- und Software-gestützten Verfahren wird diese Aufgabe meist von Logikanalysatoren erfüllt, die die an den Eingängen anliegenden Spannungen mit einer festen Frequenz erfassen und die Daten dann zum Beispiel an einen PC übertragen, an dem sie ausgewertet werden können.\\
In der vorliegenden Arbeit wird eine spezielle Form von Logikanalysator entworfen, bei der ein Teil der Auswertung bereits auf dem Logikanalysator durchgeführt wird. Das Eingangssignal wird auf bestimmte - vom Benutzer definierte - Signaländerungen untersucht und nur relevante Signaländerungen (``Events'') werden an den Benutzer weitergereicht.\\
Dieses Vorgehen bietet sich vor allem dann an, wenn der Signalverlauf grundsätzlich bekannt ist und der Fokus der Analyse auf den exakten zeitlichen Abbildung des erwarteten Signalverlaugs liegt.\\
Ein Beispiel wären die Ausgänge eines Mikrocontrollers, bei dem bewusst bestimmte Kombinationen gesetzt werden um den Start und das Ende von Funktionen im Quellcode zu signalisieren. Da das Setzen von GPIO-Pins meist in einem einzigen CPU-Takt ausgeführt werden kann, können so zuverlässige Aussagen zur Laufzeit von Funktionen, oder bei periodischer Ausführung auch zur zeitlichen Fluktuation der Funktionsausführung (``Jitter-Analyse'') getroffen werden.

\clearpage



\section{Motivation}
\label{ch:Einfuehrung:Motivation}

Die vorliegende Arbeit schließt thematisch an die Bachelorarbeit ``Ein universales, rekonfigurierbares und freies USB-Gerät zur Timing-, Protokoll-, Logik- und Eventanalyse von digitalen Signalen'' von Andreas Müller und einer darauf folgenden Projektarbeit an.\\
In der Bachelorarbeit wurde eine Hardware-Platine namens ``USB-TPLE'' mit USB-Schnittstelle, einem \acrshort{CPLD}-Chip von Altera und einem Atmega Mikrocontroller für die selbe Zielsetzung entworfen, und mit der Software-Implementierung begonnen\cite{ba:mueller}.\\  
Im nachfolgenden Semester-Projekt ``Logikanalysator mit AVR Mega32U4 und Altera MAX CPLD'' im Wintersemester 2013/14 wurde die Software-Implementierung ausgebaut und eine funktionsfähige Konfiguration für den CPLD-Chip entwickelt.\\

Im Folgenden wird allerdings ein anderer Ansatz für die Umsetzung verfolgt:
\begin{description}
\item[Verwendung von käuflich verfügbarer Hardware] \hfill \\
Anstatt der selbst entworfenen Platine soll aus Gründen der Vefügbarkeit und um die Einstiegshürde für Benutzer zu verringern ein käuflich erwerbbares Produkt verwendet werden.\\
Die Verwendung käuflicher Hardware soll außerdem die Gesamt-Komplexität des Projekts reduzieren einen stärkeren Fokus auf Grundfunktionalität ermöglichen.
\item[Verwendung eines FPGAs] \hfill \\
Um größere Flexibilität bei der Implementierung zu ermöglichen wird ein \gls{FPGA} anstatt des \gls{CPLD} verwendet (eine detailliertere Erklärung findet sich im Kapitel \nameref{ch:Design}).
\end{description}

Neben Verfügbarkeit und Flexibilität des Designs soll vor allem ein weiterer Grundsatz bei der Implementierung verfolgt werden:
\begin{description}
\item[Verwendung von Open-Source Software und Hardware] \hfill \\
Bereits die Arbeit von Andreas Müller wurde unter einer Open-Source-Lizenz veröffentlicht und es wurden alle Projekt-Quellen und Ressourcen (einschließlich des Hardwaredesigns) öffentlich verfügbar gemacht.\\
Dieser Ansatz soll hier weiter verfolgt werden, dementsprechend steht der im Rahmen dieser Arbeit entstandene Quelltext (wie große Teile der IceStorm-Toolchain) unter der \nameref{Lizenz:ISC} zur Verfügung. Die Bachelorarbeit selbst wird unter \nameref{Lizenz:CC} Lizenz veröffentlicht.\\
Ausserdem steht mit dem Projekt ``IceStorm'' erstmals auch eine Open-Source Software-Toolchain zur Programmierung von FPGA-Chips zur Verfügung, wodurch eine vollständige Open-Source Implementierung möglich wird. (In der vorliegenden Arbeit mit Ausnahme der proprietären Komponenten des Raspberry Pi Zero).
\end{description}
In Kombination mit der preisgünstigen\footnote{Der Preis des verwendeten IceZero-Boards lag zum Zeitpunkt der Erstellung der Arbeit zum Beipiel bei ca. 40€, das zur Programmierung verwendete Raspberry Pi Zero W bei unter 15€} Hardware bietet die IceStorm-Toolchain eine interessante Alternative zu den Angeboten der großen FPGA-Hersteller wie Xilinx oder Intel (ehemalig Altera), insbesondere für Lehrzwecke und kleinere Projekte.

\clearpage 
 
\section{Abgrenzung von bestehenden Lösungen zur Logikanalyse}

Es ist eine Vielzahl von kommerziellen Logikanalysatoren am Markt verfügbar. Allerdings bieten selbst flexible Geräte wie zum Beispiel die Discovery Serie von Digilent nicht die gewünschte Funktionalität der Event-Filterung zur Erfassungszeit und die Möglichkeit die so gewonnenen Daten in einen Text- bzw. Kommandozeilen-basierten Workflow einzubetten \footnote{Geräte der Discovery-Serie können durch eine \acrshort{API} z.B. in Python geskriptet werden, eine kontinuierliche ``Event-Erkennung'' scheint aber nicht ohne weiteres möglich (siehe z.B. folgender Foreneintrag\cite{forum:digilent}) }. 

Von kommerziellen Produkten abgesehen gibt es auch einige Open-Source Logikanalysatoren. Für diese Arbeit relevant sind hier vor allem:
\begin{description}
	\item \textbf{SUMP2} ist eine \gls{Verilog}-basierte Logikanalysator-Implementierung mit einer zugehörigen - in Python implementierten - grafischen Benutzeroberfläche. Es existieren angepasste Varianten von SUMP2 die ohne weitere Modifikationen auf dem auch in dieser Arbeit verwendeten iCE40-FPGA-Chip lauffähig sind\cite{web:blackmesa_sump2}.  
	\item \textbf{Open Bench Logic Sniffer} ist ein Open-Source Hardware-Produkt das auf einem Xilinx Spartan 3E FPGA basiert und eine weiterentwickelte Variante von SUMP2 verwendet. Die ``Demon core'' betitelte Weiterentwicklung ist unter anderem deshalb interessant, da mit ihr detailliertere Triggerbedingungen definiert werden können, und so zum Beispiel zeitliche und logische Abläufe von Eingangssignalen als Trigger abgebildet werden können. Hierauf soll im Kaptiel \nameref{ch:Aussicht} noch einmal eingegangen werden.
\end{description}

Das verwendete SUMP2 Datenübertragungsformat wird zum Teil auch von anderen Anwendungen unterstützt, so kann zum Beispiel der Java-Client Jawi\cite{web:ols} oder Pulseview\cite{web:sigrok_ols} (ein Qt-Frontend der libsigrok-Biliothek) als grafische Benutzeroberfläche verwendet werden.\\ 
Beide Varianten verwenden zur Datenübertragung eine serielle Schnittstelle (\acrshort{UART}), die - zumindest bei Verwendung von geläufigen Baud-Raten - die Übertragungsgeschwindigkeit stark einschränkt. Ebenso sind beide Varianten konzeptionell für die Aufnahme festgelegter und relativ kurzer Sampling-Zeiten ausgelegt und unterstützen --- wie die kommerziellen Produkte --- keine Event-Filterung zur Erfassungszeit.  

Eine Anpassung des SUMP2 Projektes wurde in Erwägung gezogen, aber aufgrund der zum Teil recht hohen Code-Komplexität und der strukturellen Unterschiede nicht durchgeführt.

\section{Zielsetzung}
\label{ch:Einfuehrung:Zielsetzung}

%Grundvoraussetzung für die Aufnahme aussagekräftiger Signaldaten ist die kontinuierliche Erfassung des Eingangssignals bei gleichbleibendem Zeitabstand. 
Für die Implementierung des Event-Recorders wurden folgende technische Ziele angestrebt:
\begin{itemize}
\item Es soll der logische Pegel von 8 bis 16 Eingangs-Pins abgefragt werden und die Eingangsdaten sollen mit einem stabilen Zeitstempel versehen werden.
\item Die zeitliche Auflösung der Aufnahme soll im Megaherz-Bereich liegen
\item Bestimmte Eingangskombinationen sollen in Textform definiert, und bei der Aufnahme als Events erkannt werden
\item Zur Steuerung der Aufnahme soll ein Kommandozeilentool zur Verfügung stehen, mit dem auch die aufgenommenen Daten in Textform abgespeichert werden können.
\end{itemize} 

\section{Aufbau der Arbeit}
\label{ch:Einfuehrung:Aufbau}

Im folgenen Kapitel \nameref{ch:Design} werden zunächst die nötigen technischen Grundlagen für die Umsetzung des Projekts besprochen, anschließend wird auf getroffene Desginentscheidungen bei der Auswahl der Hardware und Software eingegangen, und ein kurzes Implementierungs-Beispiel mit der IceStorm-Toolchain erläutert.
Das Kaptiel \nameref{ch:Implementierung} beschreibt di nötigen Anpassungen bestehender Software und die Entwicklung neuer Softwarekomponenenten bei der Durchführung des Projektes.
Im Kapitel \nameref{ch:Anwendungsfall} wird die Benutzung des Event-Recorder anhand eines konkreten Beispiels besprochen.
Es folgt ein \nameref{ch:Fazit} in dem der Status des Projekts und die Umsetzung rekapituliert werden und abschließend wird im Kapitel \nameref{ch:Aussicht} auf Optimierungsmöglichkeiten und weiteres Vorgehen eingegangen.





  
\chapter{Design}
\label{ch:Design}
Design lala \ldots

\section{Technische Grundlagen}
FPGA: datenaquise akurates timing
	- nyquist

\clearpage
\section{Benötigte Hard- und Software-Komponenten}
schneller Datenspeicher als Buffer: SRAM 
	- alternativen

Mikrocontroller / PC: datenaufarbeitung und auswertung
	außerdem: treiber/verwaltung FPGA

Kommunikation: schnittstelle (UART, SPI, I2C, etc ..)
\clearpage

\section{Auswahl der Software-Toolchain}
\clearpage

\section{Auswahl der Hardware}

FPGA: iCE40
vielzahl boards:
	z.B. https://embeddedmicro.com/products/mojo-v3
	-> SDRAM

	-> keine Opensource-Toolchain!

iCE40:
	erst:
	https://www.olimex.com/Products/FPGA/iCE40/iCE40HX1K-EVB/open-source-hardware

	dann icezero:

	- formfaktor


Controller:
	pi: preis, vielfältigkeit, formfaktor

\clearpage

\section{Beispiel: Von der Synthese bis zum Bitstream mit der IceStorm-Toolchain}
\clearpage


     
\chapter{Implementierung}
\label{ch:Implementierung}

\ldots

\clearpage

\section{Portierung des Tools zum Flashen des Bitstreams (icoprog)}
\label{ch:Implementierung:sec:icoprog}

\ldots
\clearpage

\section{Portierung und n\"otige Anpassungen des Verilog-SoCs (icosoc)}
\label{ch:Implementierung:sec:icosoc}

Bla fasel\ldots
\clearpage


\section{Implementierung des Event-Recorder Moduls}
\label{ch:Implementierung:sec:Event-Recorder}


\subsection{Bus-Schnittstelle}


\subsection{Triggerlogik}
\clearpage

\section{Implementierung eines SPI-Slave-Moduls}
\label{ch:Implementierung:sec:SPI-Slave}

Bla fasel\ldots
\clearpage
\section{Zusammenf\"uhrung der Module als Icosoc-Projekt}
\label{ch:Implementierung:sec:Icosoc-Projekt}

\section{Implementierung des textbasierten Benutzerinterfaces}
\label{ch:Implementierung:sec:Benutzerinterface}
\clearpage

\chapter{Anwendungsfall: Jitter-Analyse von Software-generierten MIDI-Clock Signalen}
\label{ch:Anwendungsfall}
Als Beispiel für ein zeitkritisches Signal sollen im Folgenden mehrere MIDI-Clock Signale mit dem Event-Recorder untersucht werden, und dabei der allgemeine Ablauf einer Event-Aufnahme und der nachfolgenden Analyse geschildert werden.
Das MIDI-Clock Signal wird in den untersuchten Fällen software-basiert -- auf einem normalen Anwender-System -- generiert, deswegen ist von einer messbaren zeitlichen Fluktuation (``Jitter'') auszugehen. Das heißt die Clock-Signale treten nicht exakt zum erwarteten Zeitpunkt sondern leicht zeitlich verfrüht oder verspätetet auf.\\
\acrshort{MIDI} ist ein 1983 spezifizierter Standard für den Austausch von Steuerinformation zwischen elektronischen Musikintrumenten und wird trotz einiger signifikanter Limitierungen seit über 35 Jahren nahezu unverändert für praktisch alle elektronischen Musikinstrumente im professionellen und im Hobby-Bereich verwendet. 
%Die wohl bedeutenste Limitierung ist die Verwendung von 7-bit-Werten für Parameter wie zum Beispiel den Notenwert, Anschlagstärke oder von Control-Change-Werten (also eine Beschränkung auf den Wertebereich von 0 -127).\\
MIDI bietet mit der ``MIDI-Clock'' eine Möglichkeit mehrere Geräte zeitlich zu synchronisieren. So könnte zum Beispiel ein Synthesizer mit der Musiksoftware auf einem PC synchronisiert werden, um tempo-abhängige Arpeggios\footnote{Ein Akkord bei dem die einzelnen Töne nicht gleichzeitig, sondern zeitversetzt nacheinander gespielt oder ausgelöst werden} zu spielen.\\
Dabei verwendet MIDI zur Datenübertragung das gleiche Protokoll wie ein \acrshort{UART} (ohne Parity-Bit) bei einer Übertragungsgeschwindigkeit von 31250 Bit/s. Die meisten MIDI-Nachrichten setzen sich aus einem Stautsbyte und zwei darauf folgenden Datenbytes zusammten, für die MIDI-Clock werden allerdings nur einzelne Bytes benötigt. \\
Zuerst wird ein Start-Byte (0xFA) übertragen, dann wird -- abhängig vom Tempo -- 24 mal pro Viertelnote ein ``Clock-Tick'' (0xF8) gesendet, und abschließend ein Stopp-Bit (0xFC). \\     
\clearpage
Eine einfache Implementierung in Python mithilfe des {\tt python-rtmidi}-Moduls könnte folgendermaßen aussehen:
\begin{minted}{python}
# MIDI CLOCK TEMPO (beats per minute)
BPM = 80

# define clock messages
clock_start = [0xFA]
clock_tick  = [0xF8]
clock_stop  = [0xFC]
[...]
    # calculate clock period
    clock_period = 60 / (BPM * 24)

    # send start byte 
    midiout.send_message(clock_start)

    # run
    while (True):
        midiout.send_message(clock_tick)
        time.sleep(clock_period)

\end{minted}

\section{Test-Setup: USB-Midi mit Teensy LC}
  
Um das MIDI-Signal auswerten zu können wird ein Mikrokontroller (``Teensy LC'') verwendet, der per USB an einen PC angeschlossen werden kann, und über die USB-Schnittstelle ein MIDI-Gerät simuliert.  Das heißt am PC wird eine virtuelle MIDI-Schnittstelle erzeugt, und alle ankommenden MIDI-Daten werden direkt zum Mikrocontroller übertragen. Der Mikrocontroller wartet auf das Start-Byte einer MIDI-Clock und setzt einen GPIO-Pin auf ``1'' um dem Event-Recorder den Anfang der Aufnahme zu signalisieren. 
Danach wird bei jedem empfangenen Clock-Tick ein zweiter GPIO-Pin kurz auf ``1'' gesetzt und anschließend wieder auf ``0''. Dies soll beim Event-Recorder als Event erkannt werden. Beim Empfangen des Stopp-Bytes wird der erste GPIO-Pin auf ``0'' gesetzt und die Aufnahme soll beendet werden.\\
Der Mikrocontroller kann mit der Arduino-IDE programmiert werden und es steht eine MIDI-Bibliothek zur Verfügung, was eine kurze und unkomplizierte Umsetzung erlaubt:
\begin{minted}{c}
[...]
// midi clock start handler (0xFA)
void onStart() {
  digitalWrite(0, HIGH);
}

// midi clock tick handler (0xF8) 
void onClock() {
    digitalWrite(1, HIGH);
    digitalWrite(1, LOW);
}

// midi clock stop handler (0xFC)
void onStop() {
  digitalWrite(0, LOW);
}

void setup() {
  // pin setup
  [...]
  // register midi handlers
  usbMIDI.setHandleStart(onStart);
  usbMIDI.setHandleStop(onStop);
  usbMIDI.setHandleClock(onClock);
}

void loop() {
  usbMIDI.read();
}
\end{minted}

\section{Einrichten des Projekts}
\label{ch:Anwendungsfall:sec:Einrichten}

Die vollständige Einrichtung des Raspberry Pi Zero W soll hier aus Platzgründen nicht beschrieben werden, es steht aber eine detailliertere Anleitung im Projekt-Wiki zur Verfügung.
Der grundsätzliche Ablauf ist wiefolgt:
\begin{enumerate}

\item Download, Kompilieren und Installation der {\tt IceStorm}-Toolchain auf dem Anwender-PC (Linux-System)

\begin{minted}{bash}
# siehe http://www.clifford.at/icestorm/#install
\end{minted}

\item Download, Kompilieren und Installation der RISC-V Toolchain

\begin{minted}{bash}
git clone git@github.com:cliffordwolf/picorv32.git
cd picorv32
make download-tools
make -j\$(nproc) build-tools
\end{minted}

Das Kompilieren der RISC-V {\tt gcc}-Toolchain kann je nach Rechenleistung mehrere Stunden dauern.

\item Download des Git-Projekts auf dem Anwender-PC (Linux-System)

\begin{minted}{bash}
cd ..
git clone https://github.com/dm7h/icozsoc.git
\end{minted}


\item Einrichten der SSH-Verbindung zum Raspberry Pi Zero W 

\begin{minted}[breaklines]{bash}
# siehe zum Beispiel: https://www.raspberrypi.org/documentation/remote-access/ssh/passwordless.md
# Exportieren des SSH-Hosts, z.B. pi@zero oder pi@192.168.1.43:
export SSH_RASPI=pi@zero
# eine SSH-Verbindung sollte dann ohne Passwort-Abfrage möglich sein:
ssh $SSH_RASPI $
\end{minted}

\item Download, Kompilieren und Installation des {\tt icozctl} Git-Projekts\\ 
Zur Installation des {\tt icozctl}-Tools wird auf dem Raspberry Pi folgendes ausgeführt:
\begin{minted}{bash}
git clone https://github.com/dm7h/icozctl
cd icozctl
sudo make install
\end{minted}

\item Generieren und Programmieren des Bitfiles mithilfe des zur Verfügung gestellten {\tt Makefiles}\\
Wenn {\tt icozctl} erfolgreich auf dem Raspberry Pi installiert wurde, kann auf dem Anwender-PC das FPGA-Bitstream und die 
IcoSoc-Anwendung kompiliert und via Raspberry Pi auf die Hardware geflasht werden:
\begin{minted}{bash}
cd ../icozsoc/examples/event-recorder/
make prog_flash
\end{minted}

 
\end{enumerate}

\section{Konfiguration der Event-Trigger}
\label{ch:Anwendungsfall:sec:Event-Trigger}

Um die Events zu konfigurieren wird zunächst per SSH auf das Raspberry Pi gewechselt.  
Im Ordner des {\tt icozctl}-Tools befindet sich die Event-Konfigurations-Datei {\tt config.yml}.
Für die Aufnahme wird nicht zwangsläufig eine Event-Erkennung benötig, es kann aber aber trotzdem ein Start-, Stopp- und Clock-Event definiert werden um unnötige Aufnahmedaten zu minimieren:

\begin{lstlisting}[language=yaml]
# event configuration
events: 
  - start:
          trigger: u
          funcion: start

  - stop:
          trigger: d
          function: stop

  - clk_up:
          trigger: 1u

  - clk_down:
          trigger: 1d
\end{lstlisting}



\section{Durchführen der Event-Aufnahme}
\label{ch:Anwendungsfall:sec:Durchführung}

Die Auswertung soll aus Anschauungszwecken das VCD-Dateiformat verwenden.
Die Aufnahme kann dann mit folgendem Befehl gestartet werden:
\begin{minted}{bash}
icozctl -c config.yml -o midi_clock.vcd
\end{minted}

Das Programm wartet nun auf Eingangsdaten, die zum Beispiel durch das Starten der in Python implementieren Midi-Clock oder durch ein Musikprogramm geliefert werden können.
Die Aufnahme kann zu jedem Zeitpunkt durch die Tastenkombination Strg+c beendet werden.
Alternativ kann {\tt icozctl} auch mit der Option ``-N'' gestartet werden, um die Aufnahme nach einer fesgelegten Anzahl von Samples zu beenden. 
\begin{minted}{bash}
icozctl -N 512 -c config.yml -o midi_clock.vcd
\end{minted}
Das Ergebnis der Aufnahme kann dann zum Beispiel mit dem Programm {\tt gtkwave} oder mit {\tt pulseview} grafisch dargestellt und überprüft werden:
\clearpage 
\begin{figure}[h]
	\centering
	\captionsetup{justification=centering,margin=2cm}
		\includegraphics[width=\textwidth]{../figures/midi_clock_pulseview.png}
		\caption[Darstellung eines aufgenommenen Midi-Clock-Signals mit Pulseview]{Darstellung eines aufgenommenen Midi-Clock-Signals mit Pulseview (Quelle: Pulseview-Screenshot)}
	\label{fig:ice40_pmod_pins}
\end{figure}


\section{Analyse der Ergebnisse}
\label{ch:Anwendungsfall:sec:Analyse}

Für die Analyse der Daten kann zum Beispiel ein Python-Skript verwendet werden. Um Aussagen über den Jitter des Signals machen zu können bietet sich die Generierung eines Histogramms an, bei dem auf der x-Achse die Abweichung zum erwarteten Clock-Signal und auf der y-Achse die Häufigkeit der Abweichung dargestellt wird.

Für das Einlesen der VCD-Datei wird das Python-Modul {\tt Verilog\_VCD} verwendet:
\begin{minted}{python}
import Verilog_VCD
vcd = Verilog_VCD.parse_vcd('midi_clock.vcd')
\end{minted}

Im VCD-Dateiformat wird jedem Signal ein Symbol als Abkürzung zugewiesen, auf die Daten des ersten Signals der Datei kann zum Beispiel mit der Abkürzung ``!'' zugegriffen werden.
Die Zeitstempel sind in der Python-Liste unter dem Index "tv" (``time value'') abrufbar. 
Der zeitliche Abstand zum jeweils vorhergehenden Event (Zeit-Delta) kann folgendermaßen berechnet werden:
 
\begin{minted}{python}
# get the first time value
last = vcd["!"]["tv"][0][0]

# calculate time deltas
deltas = []
for item in vcd["!"]["tv"][1:]: 
    deltas.append(item[0]-last)
    last = item[0]
\end{minted}

Die Daten werden in ein numpy-Array umgwandelt und es wird die Differenz zum erwarteten Zeit-Delta gebildet. 
Anschließend werden sie in eine pandas-Zeitserie konvertiert, wodurch automatisch statistisch relevante Kennwerte wie der Mittelwert und die Standardabweichung berechnet und angezeigt werden können:
 
\begin{minted}{python}
np_deltas = np.array(deltas)
expected_delta = 31250000 # expected clock period in nanoseconds
np_deltas -= expected_delta
print(deltas_series.describe())
\end{minted}

Die Ausgabe sieht zum Beispiel folgendermaßen aus\footnote{Die Werte stammen von der oben beschriebenen Python-Implementierung, das Zeit-Delta wird hier in Microsekunden-Auflösung angezeigt}

\begin{minted}{c}
count                        86
mean     0 days 00:00:00.000125
std      0 days 00:00:00.000026
min      0 days 00:00:00.000053
25%      0 days 00:00:00.000109
50%      0 days 00:00:00.000117
75%      0 days 00:00:00.000131
max      0 days 00:00:00.000227

\end{minted}

Abschließend wird mithilfe der matplotlib-Bibliothek ein Histogramm in Millisekunden-Auflösung erzeugt und angezeigt:

\begin{minted}{python}
(deltas_series/pd.Timedelta(milliseconds = 1)).hist()
[...]
plt.show()
\end{minted}

\subsection{Software MIDI-Clock: Python-Implementierung}
Für die erste Messung wurde die Python-Implementierung verwendet.\
\begin{figure}[H]
	\centering
	\captionsetup{justification=centering,margin=2cm}
		\includegraphics[width=\textwidth]{../figures/clock_python.png}
		\caption[Histogramm der Python-generierten MIDI-Clock]{Histogramm der Python-generierten MIDI-Clock}
	\label{fig:ice40_pmod_pins}
\end{figure}
Die erwartete Abweichung (0) ist mit der roten Linie markiert, der Mittlerwert mit der gestrichelten schwarzen Linie.\\
Auffällig ist hier vor allem dass die MIDI-Clock langsamer läuft als erwartet, da eine konstante positive Abweichung von ca. 0,125 Millisekunden vorliegt. Dies lässt sich dadurch erklären, dass bei jedem Ausführen des {\tt midiout.sendmessage(clock\_tick)}-Befehls zusätzlich zur berechneten Clock-Periode Zeit verbraucht wird. Der Jitter liegt bei circe 0,2 Millisekunden und ist damit vergleichsweise niedrig.


\subsection{Software MIDI-Clock: Renoise und Reaper}

Anschließend wurde die MIDI-Clock von den Musikprogrammen Renoise und Reaper aufgenommen\footnote{Es wurden (von den Vorlieben des Autors abgesehen) keine besonderen Auswahlkriterien verfolgt, beide Programme haben eine kostenlose Demo-Version mit der die Funktionalität geprüft werden kann}.

\begin{figure}[H]
	\centering
	\captionsetup{justification=centering,margin=2cm}
		\includegraphics[width=\textwidth]{../figures/clock_renoise.png}
		\caption[Histogramm der MIDI-Clock des Musikprogramms Renoise]{Histogramm der MIDI-Clock des Musikprogramms Renoise}
	\label{fig:ice40_pmod_pins}
\end{figure}

Bei der MIDI-Clock von Renoise zeigt sich ein Jitter von fast 2 Millisekunden, ein eher schlechter Wert bei dem sogar hörbare Konsequenzen auftreten könnten.

\begin{figure}[H]
	\centering
	\captionsetup{justification=centering,margin=2cm}
		\includegraphics[width=\textwidth]{../figures/clock_reaper.png}
		\caption[Histogramm der MIDI-Clock des Musikprogramms Reaper]{Histogramm der MIDI-Clock des Musikprogramms Reaper}
	\label{fig:ice40_pmod_pins}
\end{figure}

Bei der MIDI-Clock von Reaper fällt der Jitter mit circa 1,5 Millisekunden etwas geringer aus und zeigt eine starke ausgeprägte bimodale Verteilung. Auch dies ist -- inbesondere im Vergleich zur ``naiven'' Python-Implementierung -- ein eher schlechter Wert..

\subsection{Hardware MIDI-Clock: Midipal}
Abschließend wurde noch die auf einem ATMega328p-Mikrocontroller basierende Hardware-MIDI-Clock des ``MIDIpal'' getestet. Das Clock-Signal wurde über den MIDI-Eingang einer RME Digiface PCI-Soundkarte erfasst und mit einer Software-Loopback auf das virtuelle MIDI-Interface des Teensy LC geroutet.

\begin{figure}[H]
	\centering
	\captionsetup{justification=centering,margin=2cm}
		\includegraphics[width=\textwidth]{../figures/clock_midipal.png}
		\caption[Histogramm einer Mikrocontroller-generierten Hardware-MIDI-Clock]{Histogramm einer Mikrocontroller-generierten Hardware-MIDI-Clock (Mutable Instruments MidiPal)}
	\label{fig:ice40_pmod_pins}
\end{figure}

Die Messung liefert ein deutlich stabileres Ergebnis mit einem Jitter von circa 0,15 Millisekungen. Es zeigt sich, dass die Mikrokontrollerbasierte Lösung den beiden Software-Produkten in Hinblick auf die Stabilität des Clock-Signals überlegen ist.

Zu allen Messungen ist anzumerken, dass der Teensy-Mikrocontroller (der mit einer Taktfrequenz von 40 Mhz läuft) einen nicht näher bestimmten Einfluss auf das Endergebnis hat. Alle Messungen wurden wiederholt mit variierenden Samplezahlen durchgeführt, und die beschriebenen Charakteristika sind ohne nenneswerte Abweichungen reproduzierbar. Dementsprechend kann davon ausgegangen werden, dass die vom Teensy verursachte Verfälschung für die Messungen vernachlässigbar ist.



\chapter{Fazit}
\label{ch:Fazit}
In der vorliegenden Arbeit wurde das IcoSoc-Projekt auf das IceZero-FPGA-Shield portiert, und somit eine flexible Plattform für die Entwicklung von Open-Source FPGA-Projekten zur Verfügung gestellt.\\
Auf Basis der IcoSoc-Komponenten wurde ein System zur Logikanalyse entworfen, mit dem benutzerdefinierte Events in hoher zeitlicher Auflösung aufgenommen werden können.\\
Die zeitkritischeren Komponenten wurden dabei als Verilog-Module implementiert, wohingegen die Steuerung des Programmablaufs mithilfe der RISC-V CPU ``PicoRV32'' als C-Code  auf dem FPGA umgesetzt wurde.
Die Event-Erkennung wurde dabei im langsameren C-Code umgesetzt, da die Verilog-Implementierung zu viel FPGA-Ressourcen verbraucht hat und aus zeitlichen Gründen keine Optimierung mehr durchgeführt werden konnte.
Am Beispiel von verschiedenen MIDI-Clock-Signalen wurde der Ablauf einer Event-Messung erläutert und eine beispielhafte Analyse der aufgenommenen Daten durchgeführt.
Bei der Event-Erkennung und beim Datendurchsatz des Event-Recorders besteht noch Optimierungspotential.
 
\chapter{Aussicht}
\label{ch:Aussicht}

Die Implementierung der Event-Erkennung im Verilog-Teil sollte beim weiteren Vorgehen mit Priorität behandelt werden. Dies ist insbesondere für die Verwendung der eingeführten Event-Funktionen (Starten und Stoppen der Aufnahme und des Dump-Modus) von Bedeutung, da diese Funktionen im C-Teil mit deutlicher Verzögerung ausgeführt werden. Wenn zum Beispiel ein bestimmtes Event den Dump-Modus auslösen soll, um direkt darauf eine Folge von relevanten Signaländerungen zu untersuchen, werden diese Signale unter Umständen noch gar nicht erfasst, da die Funktion im C-Teil nicht rechtzeitig ausgeführt wurde.\\
Davon abgesehen würde aber auch der Gesamt-Datendurchsatz des Event-Recorders stark von einer Verilog-Implementierung profitieren.\\
Ähnliches gilt auch für den SRAM-Puffer und die Ablaufsteuerung des SPI-Transfers. Es wurden keine detaillierteren Performance-Analysen durchgeführt, es ist aber davon auszugehen, dass die Verwendung des Soft-Cores den Datendurchsatz um ein Vielfaches verringert. Es wäre also zu überlegen, ob die Funktionalität des C-Codes nicht komplett in Verilog umgesetzt wird.\\
Eine weitere Optimierung bietet sich bei der der Erfassungsgeschwindigkeit der Eingangsdaten an:
das {\tt SB\_IO}-Primitiv der Lattice Technology Library bietet eine Funktion um Eingänge sowohl bei steigender als auch bei fallender Taktflanke abzufragen und jeweils in einem eigenen Register abzulegen (siehe \cite{doc:tec_lib}). Die Erfassungsgeschwindigkeit könnte so ohne größere Code-Änderungen auf 200 Mhz verdoppelt werden.\\  
Einen Ansatz für die Optimierung der Event-Erkennung bietet das Trigger-Modul des SUMP2-Analysators (in der ``Demon core''-Version). Dabei werden die Bit-Operationen der Trigger-Logik in Blöcke von je 4 Bits Breite aufgeteilt, die dann bei der Synthese direkt als Lookup-Tabellen umgesetzt werden können (Die Optimierung ist ausführlich im Quellcode dokumentiert, siehe \cite{web:sump2_trigger}). Dieses Vorgehen sollte ohne größere Modifikationen auf die Event-Trigger angewandt werden können. \\
Davon abgesehen bietet die ``Demon core''-Version eine interessante Erweiterung für komplexere Trigger-Bedingungen. Es können mehrere Trigger-Bedingungen mit logischen Funktionen verknüpft werden und Sequenzen von Trigger-Bedingungen definiert werden (vgl. \cite{web:demon_doc}). Dies würde es zum Beispiel ermöglichen ein serielles Midi-Clock-Signal ohne einen zusätzlichen Mikrocontroller als Event zu erkennen und damit auch ohne eine mögliche zeitliche Verfälschung durch den Mikrocontroller.\\
Auch die erweiterten Trigger-Bedingungen sollten sich in den Event-Recorder integrieren lassen, und würden damit eine wertvolle Ergänzung zum bestehenden Funktionsumfang liefern. 








\appendix



%% ++++++++++++++++++++++++++++++++++++++++++
%% Abkürzungsverzeichniss
%% ++++++++++++++++++++++++++++++++++++++++++


%\printglossary[type=\acronymtype] % prints just the list of acronyms
\printglossary[type=\acronymtype]
\printglossary[type=main]

%\addcontentsline{toc}{chapter}{\protect\numberline{}{Abk\"urzungsverzeichnis}}
\clearpage

\listoffigures
\addcontentsline{toc}{chapter}{Abbildungsverzeichnis}
\clearpage

\listoftables
\addcontentsline{toc}{chapter}{Tabellenverzeichnis}
\clearpage



\ifnotonesideelse{\cleardoublepage}{}

%% ++++++++++++++++++++++++++++++++++++++++++
%% Literatur
%% ++++++++++++++++++++++++++++++++++++++++++
\addcontentsline{toc}{chapter}{\bibname}
%  mit dem Befehl \nocite werden auch nicht zitierte Referenzen abgedruckt 
% (normalerweise nicht erwünscht)
% \nocite{*}
%\bibliographystyle{plain}
%Einbinden Bibtexdatei - Direkt aus JabRef generiert
%\bibliography{literatur}
\printbibliography 

%%% Anhänge
\chapter{Material}
\label{ch:Material}
\enlargethispage{120cm}
\section{PMOD-Pinbelegung für die Event-Aufnahme}
\label{sec:pmod_pins}
Für die Event-Aufnahme sind die PMOD-Header P3 und P4 als Eingänge konfiguriert.

\begin{figure}[h]
	\centering
	\captionsetup{justification=centering,margin=2cm}
		\includegraphics[width=0.80\textwidth]{../figures/ice40_pmod_pins_01.png}
		\caption[Pinbelegung der PMOD-Header des IceZero-Boards]{Pinbelegung der PMOD-Header des IceZero-Boards}
	\label{fig:ice40_pmod_pins}
\end{figure}

Bei der Pin-Belegung wurde das selbe Schema wie beim GPIO-Modul des IcoSoc verwendet. Die Numerierung läuft dementsprechend von rechts nach links und für die einzelnen PMOD-Header ``zeilenweise'' von oben nach unten.\\
Für die Zuordnung der Signalnamen (und der zugehörigen VCD-Symbole) bei der Ausgabe im CSV- oder VCD-Format kann folgende Tabelle verwendet werden:

\begin{table}[h]
\centering
\resizebox{\columnwidth}{!}{%

\begin{tabular}{lllllllclllll}
\cline{3-6} \cline{10-13}
                                                   & \multicolumn{1}{l|}{}                                                   & \multicolumn{1}{c|}{\cellcolor[HTML]{EFEFEF},}       & \multicolumn{1}{c|}{\cellcolor[HTML]{EFEFEF}+}       & \multicolumn{1}{c|}{\cellcolor[HTML]{EFEFEF}*}       & \multicolumn{1}{c|}{\cellcolor[HTML]{EFEFEF})}       & \multicolumn{1}{c}{}  &                                                   & \multicolumn{1}{c|}{}                                                   & \multicolumn{1}{c|}{\cellcolor[HTML]{EFEFEF}\$}     & \multicolumn{1}{c|}{\cellcolor[HTML]{EFEFEF}\#}     & \multicolumn{1}{c|}{\cellcolor[HTML]{EFEFEF}"}      & \multicolumn{1}{c|}{\cellcolor[HTML]{EFEFEF}!}      \\ \cline{1-6} \cline{8-13} 
\multicolumn{1}{|l|}{\cellcolor[HTML]{DF2727}3.3V} & \multicolumn{1}{l|}{\cellcolor[HTML]{343434}{\color[HTML]{FFFFFF} GND}} & \multicolumn{1}{l|}{\cellcolor[HTML]{C0C0C0}pin\_11} & \multicolumn{1}{l|}{\cellcolor[HTML]{C0C0C0}pin\_10} & \multicolumn{1}{l|}{\cellcolor[HTML]{C0C0C0}pin\_9}  & \multicolumn{1}{l|}{\cellcolor[HTML]{C0C0C0}pin\_8}  & \multicolumn{1}{l|}{} & \multicolumn{1}{l|}{\cellcolor[HTML]{DF2727}3.3V} & \multicolumn{1}{l|}{\cellcolor[HTML]{343434}{\color[HTML]{FFFFFF} GND}} & \multicolumn{1}{l|}{\cellcolor[HTML]{C0C0C0}pin\_3} & \multicolumn{1}{l|}{\cellcolor[HTML]{C0C0C0}pin\_2} & \multicolumn{1}{l|}{\cellcolor[HTML]{C0C0C0}pin\_1} & \multicolumn{1}{l|}{\cellcolor[HTML]{C0C0C0}pin\_0} \\ \cline{1-6} \cline{8-13} 
\multicolumn{1}{|l|}{\cellcolor[HTML]{DF2727}3.3V} & \multicolumn{1}{l|}{\cellcolor[HTML]{343434}{\color[HTML]{FFFFFF} GND}} & \multicolumn{1}{l|}{\cellcolor[HTML]{C0C0C0}pin\_15} & \multicolumn{1}{l|}{\cellcolor[HTML]{C0C0C0}pin\_14} & \multicolumn{1}{l|}{\cellcolor[HTML]{C0C0C0}pin\_13} & \multicolumn{1}{l|}{\cellcolor[HTML]{C0C0C0}pin\_12} & \multicolumn{1}{l|}{} & \multicolumn{1}{l|}{\cellcolor[HTML]{DF2727}3.3V} & \multicolumn{1}{l|}{\cellcolor[HTML]{343434}{\color[HTML]{FFFFFF} GND}} & \multicolumn{1}{l|}{\cellcolor[HTML]{C0C0C0}pin\_7} & \multicolumn{1}{l|}{\cellcolor[HTML]{C0C0C0}pin\_6} & \multicolumn{1}{l|}{\cellcolor[HTML]{C0C0C0}pin\_5} & \multicolumn{1}{l|}{\cellcolor[HTML]{C0C0C0}pin\_4} \\ \cline{1-6} \cline{8-13} 
                                                   & \multicolumn{1}{l|}{}                                                   & \multicolumn{1}{c|}{\cellcolor[HTML]{EFEFEF}0}       & \multicolumn{1}{c|}{\cellcolor[HTML]{EFEFEF}/}       & \multicolumn{1}{c|}{\cellcolor[HTML]{EFEFEF}.}       & \multicolumn{1}{c|}{\cellcolor[HTML]{EFEFEF}-}       & \multicolumn{1}{c}{}  &                                                   & \multicolumn{1}{c|}{}                                                   & \multicolumn{1}{c|}{\cellcolor[HTML]{EFEFEF}(}      & \multicolumn{1}{c|}{\cellcolor[HTML]{EFEFEF}'}      & \multicolumn{1}{c|}{\cellcolor[HTML]{EFEFEF}\&}     & \multicolumn{1}{c|}{\cellcolor[HTML]{EFEFEF}\%}     \\ \cline{3-6} \cline{10-13} 
\multicolumn{6}{c}{PMOD - P4}                                                                                                                                                                                                                                                                                                                            &                       & \multicolumn{6}{c}{PMOD - P3}                                                                                                                                                                                                                                                                                                                      
\end{tabular}

%\begin{tabular}{lllllllllllll}
%\cline{1-6} \cline{8-13}
%\multicolumn{1}{|l|}{\cellcolor[HTML]{DF2727}3.3V} & \multicolumn{1}{l|}{\cellcolor[HTML]{343434}{\color[HTML]{FFFFFF} GND}} & \multicolumn{1}{l|}{pin\_11} & \multicolumn{1}{l|}{pin\_10} & \multicolumn{1}{l|}{pin\_9}  & \multicolumn{1}{l|}{pin\_8}  & \multicolumn{1}{l|}{} & \multicolumn{1}{l|}{\cellcolor[HTML]{DF2727}3.3V} & \multicolumn{1}{l|}{\cellcolor[HTML]{343434}{\color[HTML]{FFFFFF} GND}} & \multicolumn{1}{l|}{pin\_3} & \multicolumn{1}{l|}{pin\_2} & \multicolumn{1}{l|}{pin\_1} & \multicolumn{1}{l|}{pin\_0} \\ \cline{1-6} \cline{8-13} 
%\multicolumn{1}{|l|}{\cellcolor[HTML]{DF2727}3.3V} & \multicolumn{1}{l|}{\cellcolor[HTML]{343434}{\color[HTML]{FFFFFF} GND}} & \multicolumn{1}{l|}{pin\_15} & \multicolumn{1}{l|}{pin\_14} & \multicolumn{1}{l|}{pin\_13} & \multicolumn{1}{l|}{pin\_12} & \multicolumn{1}{l|}{} & \multicolumn{1}{l|}{\cellcolor[HTML]{DF2727}3.3V} & \multicolumn{1}{l|}{\cellcolor[HTML]{343434}{\color[HTML]{FFFFFF} GND}} & \multicolumn{1}{l|}{pin\_7} & \multicolumn{1}{l|}{pin\_6} & \multicolumn{1}{l|}{pin\_5} & \multicolumn{1}{l|}{pin\_4} \\ \cline{1-6} \cline{8-13} 
%\multicolumn{6}{c}{PMOD - P4}                                                                                                                                                                                                                            &                       & \multicolumn{6}{c}{PMOD - P3}                                                                                                                                                                                                                      
%\end{tabular}
}
\caption{Pinbelegung der PMOD-Header des Icezero-Boards}
\label{tbl:PMOD-Pins}
\end{table}


\clearpage
\begin{landscape}

\section{Pinverbindungen Raspberry Pi und IceZero FPGA-Shield}

Bei den mit einem ``X'' markierten Pins besteht Hardware-seitig keine Verbindung zwischen dem Raspberry Pi und dem IceZero-Board.

\begin{table}[h]
\centering
%\resizebox{\textwidth}{!}{%
\begin{tabular}{|c|c|c|c|c|
>{\columncolor[HTML]{EFEFEF}}c |
>{\columncolor[HTML]{EFEFEF}}c |c|c|c|c|c|}
\hline
\cellcolor[HTML]{C0C0C0}Funktion & \cellcolor[HTML]{C0C0C0}iCE40 & \cellcolor[HTML]{C0C0C0}IceZero & \cellcolor[HTML]{C0C0C0}WiringPi & \cellcolor[HTML]{C0C0C0}Name                        & \multicolumn{2}{c|}{\cellcolor[HTML]{C0C0C0}Physical} & \cellcolor[HTML]{C0C0C0}Name                       & \cellcolor[HTML]{C0C0C0}WiringPi & \cellcolor[HTML]{C0C0C0}IceZero & \cellcolor[HTML]{C0C0C0}iCE40 & \cellcolor[HTML]{C0C0C0}Funktion \\ \hline
                                 &                               &                                 &                                  & \cellcolor[HTML]{DF2727}3.3V                        & 1                         & 2                         & \cellcolor[HTML]{DF2727}5V                         &                                  &                                 &                               &                                  \\ \hline
unused                           & 115                           & PI\_I2C\_SDA                    & 8                                & SDA.1                                               & 3                         & 4                         & \cellcolor[HTML]{DF2727}5V                         &                                  &                                 &                               &                                  \\ \hline
unused                           & 114                           & PI\_I2C\_SCL                    & 9                                & SCL.1                                               & 5                         & 6                         & \cellcolor[HTML]{333333}{\color[HTML]{FFFFFF} GND} &                                  &                                 &                               &                                  \\ \hline
                                 &                               & {\color[HTML]{000000} X}        & 7                                & 1-Wire                                              & 7                         & 8                         & TxD                                                & 15                               & PI\_UART\_WI                    & 113                           & DEBUG\_UART                      \\ \hline
                                 &                               &                                 &                                  & \cellcolor[HTML]{333333}{\color[HTML]{FFFFFF} GND}  & 9                         & 10                        & RxD                                                & 16                               & PI\_UART\_RO                    & 112                           & DEBUG\_UART                      \\ \hline
                                 &                               & X                               & 0                                & GPIO.0                                              & 11                        & 12                        & GPIO.1                                             & 1                                & X                               &                               &                                  \\ \hline
                                 &                               & X                               & 2                                & GPIO.2                                              & 13                        & 14                        & \cellcolor[HTML]{333333}{\color[HTML]{FFFFFF} GND} &                                  &                                 &                               &                                  \\ \hline
unused                           & 101                           & PI\_GPIO\_2                     & 3                                & GPIO.3                                              & 15                        & 16                        & GPIO.4                                             & 4                                & X                               &                               &                                  \\ \hline
                                 &                               &                                 &                                  & \cellcolor[HTML]{DF2727}{\color[HTML]{000000} 3.3V} & 17                        & 18                        & GPIO.5                                             & 5                                & PI\_GPIO\_1                     & 99                            & unused                           \\ \hline
DATA\_MOSI                       & 90                            & PI\_SPI\_MOSI                   & 12                               & MOSI                                                & 19                        & 20                        & \cellcolor[HTML]{333333}{\color[HTML]{FFFFFF} GND} &                                  &                                 &                               &                                  \\ \hline
DATA\_MISO                       & 87                            & PI\_SPI\_MISO                   & 13                               & MISO                                                & 21                        & 22                        & GPIO.6                                             & 6                                & PI\_GPIO\_0                     & 88                            & unused                           \\ \hline
DATA\_SCK                        & 79                            & PI\_SPI\_SCK                    & 14                               & SCLK                                                & 23                        & 24                        & CE0                                                & 10                               & PI\_SPI\_CE\_0                  & 85                            & DATA\_SS                         \\ \hline
                                 &                               &                                 &                                  & \cellcolor[HTML]{333333}{\color[HTML]{FFFFFF} GND}  & 25                        & 26                        & CE1                                                & 11                               & PI\_SPI\_CE\_1                  & 78                            & unused                           \\ \hline
unused                           & 73                            & PI\_ID\_0                       & 30                               & SDA.0                                               & 27                        & 28                        & SCL.0                                              & 31                               & PI\_ID\_1                       & 74                            & unused                           \\ \hline
CONFIG\_DONE                     & 65                            & CFG\_DONE                       & 21                               & GPIO.21                                             & 29                        & 30                        & \cellcolor[HTML]{333333}{\color[HTML]{FFFFFF} GND} &                                  &                                 &                               &                                  \\ \hline
CONFIG\_MOSI                     & 68                            & CFG\_SI                         & 22                               & GPIO.22                                             & 31                        & 32                        & GPIO.26                                            & 26                               & CFG\_SS                         & 71                            & CONFIG\_SS                       \\ \hline
CONFIG\_MISO                     & 67                            & CFG\_SO                         & 23                               & GPIO.23                                             & 33                        & 34                        & \cellcolor[HTML]{333333}{\color[HTML]{FFFFFF} GND} &                                  &                                 &                               &                                  \\ \hline
                                 &                               & X                               & 24                               & GPIO.24                                             & 35                        & 36                        & GPIO.27                                            & 27                               & CFG\_SCK                        & 70                            & CONFIG\_SCK                      \\ \hline
CONFIG\_RESET                    & 66                            & CFG\_RST\_1                     & 25                               & GPIO.25                                             & 37                        & 38                        & GPIO.28                                            & 28                               & X                               &                               &                                  \\ \hline
                                 &                               &                                 &                                  & \cellcolor[HTML]{333333}{\color[HTML]{FFFFFF} GND}  & 39                        & 40                        & GPIO.29                                            & 29                               & X                               &                               &                                  \\ \hline
\end{tabular}%
%}
\caption{Pinverbindungen Raspberry Pi und IceZero FPGA-Shield}
\label{tbl:pins}
\end{table}
\end{landscape}

\clearpage
\section{Detaillierte Pinbelegung aller PMOD-Header}
\label{sec:pmod_all}

%PMOD1
\begin{table}[H]
\centering
\caption{Pinbelegung PMOD-P1}
\label{tbl:pmod1}
\begin{tabular}{|l|l|l|l|l|}
\hline
\cellcolor[HTML]{EFEFEF}130       & \cellcolor[HTML]{EFEFEF}135      & \cellcolor[HTML]{EFEFEF}137      & \cellcolor[HTML]{EFEFEF}139      & iCE40 PCF Pin \\ \hline
GPIO\_PIN\_7                      & GPIO\_PIN\_5                     & GPIO\_PIN\_3                     & GPIO\_PIN\_1                     & IceZero Name  \\ \hline
\cellcolor[HTML]{C0C0C0}pmod1\_4  & \cellcolor[HTML]{C0C0C0}pmod1\_3 & \cellcolor[HTML]{C0C0C0}pmod1\_2 & \cellcolor[HTML]{C0C0C0}pmod1\_1 & IcoSoc Name   \\ \hline
\cellcolor[HTML]{C0C0C0}pmod1\_10 & \cellcolor[HTML]{C0C0C0}pmod1\_9 & \cellcolor[HTML]{C0C0C0}pmod1\_8 & \cellcolor[HTML]{C0C0C0}pmod1\_7 & IcoSoc Name   \\ \hline
GPIO\_PIN\_6                      & GPIO\_PIN\_4                     & GPIO\_PIN\_2                     & GPIO\_PIN\_0                     & IceZero Name  \\ \hline
\cellcolor[HTML]{EFEFEF}134       & \cellcolor[HTML]{EFEFEF}136      & \cellcolor[HTML]{EFEFEF}138      & \cellcolor[HTML]{EFEFEF}141      & iCE40 PCF Pin \\ \hline
\multicolumn{5}{|c|}{PMOD P1}                                                                                                                              \\ \hline
\end{tabular}
\end{table}

%PMOD2
\begin{table}[H]
\centering
\caption{Pinbelegung PMOD-P2}
\label{tbl:pmod2}
\begin{tabular}{|l|l|l|l|l|}
\hline
\cellcolor[HTML]{EFEFEF}43        & \cellcolor[HTML]{EFEFEF}45       & \cellcolor[HTML]{EFEFEF}48       & \cellcolor[HTML]{EFEFEF}56       & iCE40 PCF Pin \\ \hline
GPIO\_PIN\_15                     & GPIO\_PIN\_13                    & GPIO\_PIN\_11                    & GPIO\_PIN\_9                     & IceZero Name  \\ \hline
\cellcolor[HTML]{C0C0C0}pmod2\_4  & \cellcolor[HTML]{C0C0C0}pmod2\_3 & \cellcolor[HTML]{C0C0C0}pmod2\_2 & \cellcolor[HTML]{C0C0C0}pmod2\_1 & IcoSoc Name   \\ \hline
\cellcolor[HTML]{C0C0C0}pmod2\_10 & \cellcolor[HTML]{C0C0C0}pmod2\_9 & \cellcolor[HTML]{C0C0C0}pmod2\_8 & \cellcolor[HTML]{C0C0C0}pmod2\_7 & IcoSoc Name   \\ \hline
GPIO\_PIN\_14                     & GPIO\_PIN\_12                    & GPIO\_PIN\_10                    & GPIO\_PIN\_8                     & IceZero Name  \\ \hline
\cellcolor[HTML]{EFEFEF}42        & \cellcolor[HTML]{EFEFEF}44       & \cellcolor[HTML]{EFEFEF}47       & \cellcolor[HTML]{EFEFEF}55       & iCE40 PCF Pin \\ \hline
\multicolumn{5}{|c|}{PMOD P2}                                                                                                                              \\ \hline
\end{tabular}
\end{table}

%PMOD3

\begin{table}[H]
\centering
\caption{Pinbelegung PMOD-P3}
\label{tbl:pmod3}
\begin{tabular}{|l|l|l|l|l|}
\hline
\cellcolor[HTML]{EFEFEF}52        & \cellcolor[HTML]{EFEFEF}28       & \cellcolor[HTML]{EFEFEF}29       & \cellcolor[HTML]{EFEFEF}26       & iCE40 PCF Pin \\ \hline
GPIO\_PIN\_31                     & GPIO\_PIN\_30                    & GPIO\_PIN\_29                    & GPIO\_PIN\_28                    & IceZero Name  \\ \hline
\cellcolor[HTML]{C0C0C0}pmod3\_4  & \cellcolor[HTML]{C0C0C0}pmod3\_3 & \cellcolor[HTML]{C0C0C0}pmod3\_2 & \cellcolor[HTML]{C0C0C0}pmod3\_1 & IcoSoc Name   \\ \hline
\cellcolor[HTML]{C0C0C0}pmod3\_10 & \cellcolor[HTML]{C0C0C0}pmod3\_9 & \cellcolor[HTML]{C0C0C0}pmod3\_8 & \cellcolor[HTML]{C0C0C0}pmod3\_7 & IcoSoc Name   \\ \hline
GPIO\_PIN\_19                     & GPIO\_PIN\_18                    & GPIO\_PIN\_17                    & GPIO\_PIN\_16                    & IceZero Name  \\ \hline
\cellcolor[HTML]{EFEFEF}37        & \cellcolor[HTML]{EFEFEF}38       & \cellcolor[HTML]{EFEFEF}39       & \cellcolor[HTML]{EFEFEF}41       & iCE40 PCF Pin \\ \hline
\multicolumn{5}{|c|}{PMOD P3}                                                                                                                              \\ \hline
\end{tabular}
\end{table}

%PMOD4

\begin{table}[H]
\centering
\caption{Pinbelegung PMOD-P4}
\label{tbl:pmod4}
\begin{tabular}{|l|l|l|l|l|}
\hline
\cellcolor[HTML]{EFEFEF}7         & \cellcolor[HTML]{EFEFEF}8        & \cellcolor[HTML]{EFEFEF}20       & \cellcolor[HTML]{EFEFEF}21       & iCE40 PCF Pin      \\ \hline
GPIO\_PIN\_27                     & GPIO\_PIN\_26                    & GPIO\_PIN\_25                    & GPIO\_PIN\_24                    & IceZero Name \\ \hline
\cellcolor[HTML]{C0C0C0}pmod4\_4  & \cellcolor[HTML]{C0C0C0}pmod4\_3 & \cellcolor[HTML]{C0C0C0}pmod4\_2 & \cellcolor[HTML]{C0C0C0}pmod4\_1 & IcoSoc Name  \\ \hline
\cellcolor[HTML]{C0C0C0}pmod4\_10 & \cellcolor[HTML]{C0C0C0}pmod4\_9 & \cellcolor[HTML]{C0C0C0}pmod4\_8 & \cellcolor[HTML]{C0C0C0}pmod4\_7 & IcoSoc Name  \\ \hline
GPIO\_PIN\_23                     & GPIO\_PIN\_22                    & GPIO\_PIN\_21                    & GPIO\_PIN\_20                    & IceZero Name \\ \hline
\cellcolor[HTML]{EFEFEF}142       & \cellcolor[HTML]{EFEFEF}143      & \cellcolor[HTML]{EFEFEF}144      & \cellcolor[HTML]{EFEFEF}1        & iCE40 PCF Pin      \\ \hline
\multicolumn{5}{|c|}{PMOD P4}                                                                                                                             \\ \hline
\end{tabular}
\end{table}

\clearpage

\section{Inhalt der CD}
Die beiliegende CD enthält den Inhalt des Github-Repositories 
\begin{lstlisting}[language=bash]
https://github.com/dm7h/fpga-event-recorder
\end{lstlisting}
zum Zeitpunkt der Abgabe.

Die folgende Tabelle bietet einen Überblick über die Inhalte des Repositories:
\begin{table}[h]
\centering

\begin{tabular}{|p{1cm}|p{3cm}|p{10cm}|}
\hline
\rowcolor[HTML]{C0C0C0} 
Verz. & Unterverzeichnis                & Inhaltsbeschreibung                                                                                                                                                                                                                                                                                                      \\ \hline
thesis/     &                                 & Finale Version der Bachelorarbeit als PDF                                                                                                                                                                                                                                                             \\ \hline
thesis/     & src/                            & Latex-Sourcen der Bachelorarbeit                                                                                                                                                                                                                                                              \\ \hline
doc/        & ref/                            & iCE40 Manuals und relevante Hardwaredokumentation                                                                                                                                                                                                                                                                        \\ \hline
misc/       &                                 & Nicht direkt Event-Recorder bezogene Informationen und  Code (u.a. für die Analyse)                                                                                                                                                                                                                                                             \\ \hline
src/        &                                 &                                                                                                                                                                                                                                                                                                                          \\ \hline
            & Logikanalysator/                & Dateien des Semesterprojekts ``Logikanalysator mit AVR Mega32U4 und Altera MAX CPLD''  inkl. Dateien der Bachelorarbeit von Andreas Müller (USB-TPLE)  
															     \\ \hline
            & icotools/                       & Portierung des icotools Projekts für das IceZero-Board. Original-Repository: https://github.com/cliffordwolf/icotools. Relevant sind hier vor allem die Unterverzeichnisse ``icosoc'' und ``icozprog'' 
\\ \hline
            & icozctl/                        & Fork des icoprog-Tools (icotools/icoporg) mit zusätzlichen Funktionen zur Steuerung des Event-Recorders                                                                                                                                                                                                                  \\ \hline
            & picosoc/                        & Portierung des PicoSoc-Projekts auf das IceZero-Board (nicht direkt projekt-relevant). PicoSoc ist eine minimale Variante des IcoSocs. Original-Repository: https://github.com/cliffordwolf/picorv32/tree/master/picosoc                                                    \\ \hline
            & sump2\_ice40/                          & SUMP2 Variante für das IceZero-Board. Quelle: https://blackmesalabs.wordpress.com/2017/02/07/icezero-fpga-board-for-rasppi/                                                                                                                                                                                                                                                    \\ \hline
            & sump2\_pipistrello\_ ftdi\_fifo/ & SUMP2 Vairante für das Pipistrello-Board bei dem der UART durch einen FTDI-Fifo ersetzt wurde, um einen Höheren Datenduchsatz zu ermöglichen. Quelle: http://forum.gadgetfactory.net/topic/1748-open-bench-logic-sniffer-with-64mb-capture-buffer/ \\ \hline
            & demon-core-import/              & SUMP2 Weiterenwicklung für den Open Bench Logic Sniffer. Quelle: https://github.com/jhol/demon-core-import                                                                                                                                                                     \\ \hline
\end{tabular}
\caption{Überblick über den Inhalt des Git-Repositories}
\label{tbl:git_repo}
\end{table}

\chapter{GPL}
\label{ch:GPL}
%% ==============================
Anhang B \ldots




%% ++++++++++++++++++++++++++++++++++++++++++
%% Index (optional)
%% ++++++++++++++++++++++++++++++++++++++++++
%\ifnotdraft{
\addcontentsline{toc}{chapter}{Index}
\printindex            % Index, Stichwortverzeichnis
%}
\end{document}
