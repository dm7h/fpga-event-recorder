
\newdualentry{FPGA}
	{FPGA}
	{Field Programmable Gate Array}
	{Ein rekonfigurierbarer Chip dessen Schaltungsstruktur in einer Hardwarebeschreibungssprache (wie VHDL oder Verilog) frei programmierbar ist\cite{wiki:FPGA}} 

\newdualentry{CPLD}
	{CPLD}
	{Complex Programmable Logic Device}
	{Im Vergleich zu FPGAs deutlich einfacher Aufgebaute programmierbare logische Schaltungen\cite{wiki:CPLD}}

\newdualentry{VHDL}
	{VHDL}
	{Very High Speed Integrated Circuit Hardware Description Language}
	{Vor allem in Europa verbeitetete Hardwarebeschreibungssprache für die Simulation und Programmierung von FPGAs und CPLDs\cite{wiki:VHDL}}


\newdualentry{API}
	{API}
	{Application Programming Interface}
	{Schnittstelle zur Anwendungsprogrammierung. Erlaubt zum Beispiel die Verwendung von Programmkomponeneten durch eine exteren Skript-Sprache wie Python\cite{wiki:API}}

\newdualentry{UART}
	{UART}
	{Universal Asynchronous Receiver Transmitter}
	{Schnittstelle zur asynchronen, seriellen Datenübertragung (vgl. \cite{wiki:UART})}


\newdualentry{SPI}
	{SPI}
	{Serial Peripheral Interface}
	{Synchroner, serieller Datenbus für Datenübertragungen nach dem Master-Slave-Prinzip mit dem vergleichsweise hohe Datendurchsätze möglich sind (vgl. \cite{wiki:SPI})}





\newglossaryentry{Verilog}{
	name=Verilog,
	description={Wortkreuzung aus "Verification" und "Logic". Hardwarebeschreibungssparche für Programmierung und Simulation von FPGAs und CPLDs die in den USA geläufiger ist als VHDL (\cite{wiki:Verilog}, siehe auch \cite{pdf:golson} zur Namesnherkunft)}}


