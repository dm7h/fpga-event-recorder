
\newdualentry{FPGA}
	{FPGA}
	{Field Programmable Gate Array}
	{Ein rekonfigurierbarer Chip dessen Schaltungsstruktur in einer Hardwarebeschreibungssprache (wie VHDL oder Verilog) frei programmierbar ist (vgl. \cite{wiki:FPGA})} 

\newdualentry{CPLD}
	{CPLD}
	{Complex Programmable Logic Device}
	{Im Vergleich zu FPGAs einfacher aufgebaute programmierbare logische Schaltungen (vgl. \cite{wiki:CPLD})}

\newdualentry{VHDL}
	{VHDL}
	{Very High Speed Integrated Circuit Hardware Description Language}
	{Vor allem in Europa verbeitetete Hardwarebeschreibungssprache für die Simulation und Programmierung von FPGAs und CPLDs (vgl. \cite{wiki:VHDL})}


\newdualentry{API}
	{API}
	{Application Programming Interface}
	{Schnittstelle zur Anwendungsprogrammierung. Erlaubt zum Beispiel die Verwendung von Programmkomponeneten durch eine exteren Skript-Sprache wie Python (vgl. \cite{wiki:API})}

\newdualentry{UART}
	{UART}
	{Universal Asynchronous Receiver Transmitter}
	{Schnittstelle zur asynchronen, seriellen Datenübertragung (vgl. \cite{wiki:UART})}


\newdualentry{SPI}
	{SPI}
	{Serial Peripheral Interface}
	{Synchroner, serieller Datenbus für Datenübertragungen nach dem Master-Slave-Prinzip mit dem vergleichsweise hohe Datendurchsätze möglich sind (vgl. \cite{wiki:SPI})}


\newdualentry{PLL}
	{PLL}
	{Phase-locked loop}
	{Elektronische Schaltungsanordnung die die Frequenz eines veränderbaren Oszillators beeinflussen kann. Wird bei FPGAs meist zur Erzeugung von Taktsignalen mit einstellbarer Geschwindkeit verwendet (vgl. \cite{wiki:PLL})}

\newdualentry{I2C}
	{$\text{I}^2$C}
	{Inter-Integrated Circuit}
	{Synchroner, serieller Datenbus nach dem Master-Slave-Prinzip, der Übertragunsraten bis zu 5 Mbit/s ermöglicht (vgl. \cite{wiki:I2C})}

\newdualentry{GPIO}
	{GPIO}
	{General Purpose Input / Output}
	{Frei programmierbarer Kontakt der zum Beispiel für das Ansprechen externer Hardwarekomponenten verwendet werden kann (vgl. \cite{wiki:GPIO})}


\newdualentry{VCD}
	{VCD}
	{Value}
	{Text-basiertes Dateiformat zur Speicherung von zeitlichen Signalabfolgen  (vgl. \cite{wiki:VCD})}


\newacronym{RAM}
	{RAM}
	{Random-Access Memory} 

\newacronym{MIDI}
	{MIDI}
	{Musical Instrument Digital Interface} 

\newglossaryentry{PMOD}{
	name=Pmod,
	description={Offener Standard für eine Mikrokontroller- und FPGA-Schnittstelle bei der 6 Pins mit einer Masse-, Stromleitung und 4 Datenleitgungen zusammengefasst werden}}

\newglossaryentry{Reverse-Engineering}{
	name=Reverse Engineering,
	description={"Reverse Engineering [...] bezeichnet den Vorgang, aus einem bestehenden fertigen System oder einem meistens industriell gefertigten Produkt durch Untersuchung der Strukturen, Zustände und Verhaltensweisen die Konstruktionselemente zu extrahieren."\cite{wiki:reverse_eng}}}


\newglossaryentry{Verilog}{
	name=Verilog,
	description={Wortkreuzung aus "Verification" und "Logic". Hardwarebeschreibungssparche für Programmierung und Simulation von FPGAs und CPLDs die in den USA geläufiger ist als VHDL (vgl. \cite{wiki:Verilog}, siehe auch \cite{pdf:golson} zur Namesnherkunft)}}




