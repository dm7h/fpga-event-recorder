\chapter{Einf�hrung} % Kapitel 1

\section{Motivation} \label{Motivation}

Digitale Schaltungen und Mikroprozessoren befinden sich heutzutage in so gut wie jedem Ger�t. Die Bandbreite geht hier von einfachen Kaffemaschinen, bis hin zur komplexen Automatiserungstechnik zum Beispiel einer Autowaschstra�e.

Hierf�r sind auch zeitlich miteinander abgestimmte Prozesse und Echtzeitanwendungen n�tig. So l�st zum Beispiel ein Sensor, wie eine Lichtschranke, einen Interrupt-Subroutine innerhalb eines Mikrocontrollers aus. Dieser springt nun aus seinem Hauptprogramm in das entsprechende Unterprogramm. Nun ist es f�r den problemlosen Ablauf einer Steuerung manchmal wichtig, schon vor Ausf�hrung des Unterprogramms zu wissen, wie lange die Ausf�hrung vermutlich dauern wird.

Dies kann nur durch ein vorheriges messen der Dauer erfolgen. Diese Messung kann nun nat�rlich von der Anwendung selbst erfolgen. Jedoch muss hier beachtet werden, dass die Funktion zur Zeitmessung ebenfalls Ausf�hrungszeit ben�tigt, was das Ergebnis, welches im Millisekunden-Bereich liegen kann, verf�lscht. 

Als weiteres kommt noch eine externe Messung in Frage. Daf�r wird zu Beginn und Ende der zu messenden Funktion ein Hardwarpin gesetzt. Dieses Setzen eines Hardwarpins verf�lscht bei modernen Mikrocontrollern das Ergebnis maximal um einen Takt. Nun kann dieses Signal durch ein externes Ger�t registriert werden, und die Zeit zwischen den beiden Pulsen gemessen werden. 

Ein solches Ger�t m�sste mindestens �ber einen Messleitungs-Eingang, einen Zeitgeber und einen Zwischenspeicher verf�gen. Auch ein Ausgang zur Datenanalyse muss vorhanden sein.

\section{Ziel der Arbeit}

Ziel der Arbeit ist es, eine Basis f�r den oben erw�hnen Analysator zu schaffen. Dazu geh�ren die Auswahl der Komponenten, das Erstellen des Schaltplanes und das Fertigen und Testen des Prototypen.

Auf Softwareseite sollen M�glichkeiten aufgezeigt werden, wie mit dem Ger�t kommuniziert werden kann. Dazu geh�ren die Implementierung einer USB-Schnittstelle f�r die Kommunikation mit der Hardware. Ausserdem soll �ber die selbe Schnittstelle ein Austausch der Firmware erfolgen k�nnen.

Ein komplett funktionsf�higer Analysator wurde in dieser Arbeit nicht erstellt. Jedoch wird eine m�gliche Implementierung sowohl auf Softwareseite als auch auf Seite der logischen Schaltung erl�utert.

Auf der so erstellten Basis, kann nun in zuk�nftigen Arbeiten oder Projektgruppen ein Analysator entwickelt werden, der die unter Abschnitt \ref{Motivation} erw�hnten Funktionen erm�glicht. Auch sind durch die Flexibilit�t des verwendeten Mikrocontrollers und des Logikbausteins eine Vielzahl weiterer Funktionen m�glich.

\section{Open-Source}

Die gesamte Arbeit steht unter einer Open-Source Lizenz. Dies betrifft zum einen die Software, also die Firmware des Mikrocontrollers, den VHDL-Code des Logikbausteins und die PC-Software. Au�erdem steht die Hardware, also die Schaltpl�ne und das Platinenlayout, auch unter einer Open-Hardware-Lizenz.

Als Lizenz wurde die LGPL-Lizenz ausgew�hlt. Diese Lizenz erlaubt es auch komerziellen Anwendern die Soft- und Hardware in Ihre Anwendungen zu Integrieren. Eine genaue Beschreibung der Lizenz befindet sich im Anhang. Einige der verwendeten Codeabschnitte und Programme verf�gen �ber eine eigene Lizenz. Der Hinweis auf diese Lizenzen befindet sich in den entsprechenden Quellcode-Dateien. 

F�r den schriftlichen Teil der Bachelorarbeit wurde die GNU-Lizenz f�r freie Dokumentation ausgew�hlt, was sowohl nichtkommerzielles als auch kommerzielles Kopieren ausdr�cklich erlaubt.

\begin{center}
\includegraphics[width=0.3\textwidth]{images/lgpl.png}
\end{center}

\section{Aufbau der Arbeit}

In Kapitel 2 wird die Theorie der digitalen Messtechnik angeschnitten. In den Kapiteln 3 und 4 wird die Entwicklung des Hardwareprototypen erl�utert. Die Kapitel 5 bis 7 befassen sich mit der Firmware des Mikrocontrollers. Die PC-Software wird in Kapitel 8 und da VHDL-Desighn in Kapitel 9 erkl�rt. 

Ein Ausblick auf die zuk�nftige Entwicklung befindet sich unter Kapitel 10. Im Anhang sind Schaltpl�ne, Listen und Verzeichnisse sowie die LGPL-Lizenz abgedruckt.