% ==========================================================
% Dokumentenart einstellen
% ==========================================================
%\documentclass[%
%10pt,			% 10 Pt Schriftgr??e [default]
%12pt, 			% 12 Pt Schriftgr??e
%onecolumn,	% Eine Spalte [default]
%oneside,		% Einseitiger Druck,
%twosided,		% zweiseitiger Druck, funktioniert nur bei book
%a4paper,		% Format: A4
%pdftex,			% PDF-Erzeugung erzwingen
%titlepage,		% Titelseite vorhanden
%openright		% funktioniert nur bei Report oder book
%]{report} 	% sonstige Einstellungen aus der Dokumentenart "report" verwenden

		
%\usepackage[nottoc]{tocbibind} % Literaturverzeichnis, Index, ... automatisch
															 % in die Inhaltsangabe schreiben
															 % im Normalfall muss dies mit dem Befehl
															 % \addcontentsline{toc}{chapter}{Name} geschehen
		

% ==========================================================
% Spracheinstellungen
% ==========================================================

\usepackage[ngerman]{babel}	% Silbentrennung nach neuer deutscher 
														% Rechtschreibung erm?glichen
\usepackage{bibgerm}		% Deutsches Literaturverzeichnis, nur n?tig wenn "thebibliography nicht reicht"
\usepackage{ae}					% Textsatz u.a. in Formeln optimieren
\usepackage[right]{eurosym}		% Euro Zeichen (laden erforderlich, da nicht in Standardzeichensatz vorhanden)
															% Eurozeichen recht vom Betrag wenn mit \EUR{2,49} aufgerufen sonst [left] nehmen

% ==========================================================
%% Mathematische Symbole
% ==========================================================
\usepackage{amsmath} 	% Formelsatz erm?glichen
\usepackage{amstext}	% Text in Formeln per \text{hier kommt der Text}
\usepackage{amsfonts} % f?r komplexere Formeln mit z.B. Mengensymbolen
\usepackage{amssymb}  % f?r komplexere Formeln mit z.B. Mengensymbolen
\usepackage{lscape} %Erm�glicht Querformatseiten

% ==========================================================
% Quelltext 
% ==========================================================
\usepackage{listings} % Paket f?r Syntax Highlighting in Quelltext
\lstset{
numbers=left, 			% Zeilennummerierung links
%numberstyle=\tiny, % kleine Nummern
numbersep=5pt,			% Abstand der Zeilennummerierung
breaklines=true, 		% Zeilenumbr?che erlauben
captionpos=b,			% Beschriftung unter [bottom] dem Quelltext
frame=lines, % Oberhalb und unterhalb des Listings ist eine Linie
basicstyle=\small \ttfamily, % Schriftart
keywordstyle=\color{blue}, % Farbe f�r die Keywords wie public, void, object u.s.w.
commentstyle=\color{kotzgruen}, % Farbe der Kommentare
stringstyle=\color{red}, % Farbe der Zeichenketten
numbersep=5pt,
breaklines=true, % Wordwrap a.k.a. Zeilenumbruch aktiviert
showstringspaces=false,
% emph legt Farben f�r bestimmte W�rter manuell fest
emph={double,bool,int,unsigned,char,true,false,void},
emphstyle=\color{blue},
emph={Assert,Test},
emphstyle=\color{red},
emph={[2]\using,\#define,\#ifdef,\#endif}, emphstyle={[2]\color{blue}}
} 
		
\lstset{language=C} % gew?nschte Programmiersprache, kann direkt beim Aufruf auch per
											% \begin[language=xxx]{lstlisting} gesetzt werden
											% Sprachen sind u.a. VBScript, C++, PHP, TeX, Java, SQL, XML

% ==========================================================
% Tastatur-Kodierung (PC/MAC)
% ==========================================================
%\usepackage[applemac]{inputenc}	% MAC
\usepackage[latin9]{inputenc}			% PC
\usepackage[T1]{fontenc}					% Umlaute richtige trennen

% Celsius
\usepackage{textcomp}							% Paket f?r Sonderzeichen


% ==========================================================
% Bilder, Tabellen und PDFs(einbinden)
% ==========================================================
\usepackage{graphicx}	%macht Einbinden von Bildern und Grafiken m?glich
\usepackage{float} 	% Bietet die Option H zum wirklich 
										% festen verankern von Elementen
\usepackage{rotating} % Jedes Objekt rotieren. Im Text sind dann folgende
											%  Umgebungen vorhanden:
											% \begin{sideways} % Obj. um 90? drehen
											% \begin{turn}{30} % Obj. um 30? drehen
											% \begin{rotate}{30} % wie turn aber mit mehr Abstand
											%\begin{sidewaysfigure} % gleitendes Obj. auf eigene Seite
																							% und um 90? gedreht
											% \begin{sidewaystable} % siehe sidewaysfigure
								
\usepackage{multirow}	% komplexere Tabellen erm?glichen
											% http://andrewjpage.com/index.php?/archives/43-Multirow-and-multicolumn-spanning-with-latex-tables.html
\usepackage{array}	% Macht Einbinden von Tabellen m?glich
\usepackage{floatflt} % Macht Textfluss link oder recht neben einem Bild
											% \begin{floatingfigure}[r]{0.4\textwidth}
											%	% r = Bild rechts, l = bild links
											%   \centering
											%   \includegraphics[width=0.30\textwidth]{res/meinbild.png}
											%   \caption{Das ist mein Bild}
											%   \label{fig:Mein Bild}
											% \end{floatingfigure}
																	
\usepackage{longtable} %Erm?glicht Tabellen die l?nger als eine Seite sind
\usepackage{pdfpages}	% Erm?glicht einbinden von externen PDFs

% ==========================================================
% Farben
% ==========================================================
\usepackage{color} % Eigene Farben k?nnen definiert werden
\definecolor{kotzgruen}{rgb}{0.0,0.5,0.0}
\definecolor{darkred}{rgb}{0.7,0.0,0.0}

% ==========================================================
% Sonstiges
% ==========================================================
\usepackage{pslatex}	% 'standard' PS Fonts verwenden
\usepackage{pifont} 	% spezielle Sonderzeichen erm?glichen
											% http://willbenton.com/wb-images/pifont.pdf
\usepackage{multicol} % Zum verwenden mehrerer Spalten innerhalb einer Umgebung


% ==========================================================
% Papierformat und Seitenr?der auf DIN-A4 einstellen
% ==========================================================
\usepackage{setspace}
\usepackage[paper=a4paper,left=25mm,right=25mm,top=30mm,bottom=30mm]{geometry}

% ==========================================================
% Kopfzeile: Name der Section, Fusszeile: rechtsb?ndig ``Seite von Gesamtseitenzahl''
% ==========================================================
\usepackage{fancyhdr} 			% laden des pakets
\pagestyle{fancy}			% festlegen des pagestyle
%
%%Kopf- und Fusszeilendefinition f?r normalen Flie?text
%\fancyhead{}
%\fancyfoot{}
\fancyhf{}
\lhead{\leftmark}
\rfoot{\thepage}
\renewcommand{\headrulewidth}{0.4pt} % Linie mit 0.4pt oben
\renewcommand{\footrulewidth}{0.4pt} % Linie mit 0.4pt unten

%%Kopf- und Fusszeilendefinition f?r spezielle Seiten wie z.B Chapter 
%%% Diese Seiten verwenden den Pagestyle plain, welcher nun im folgenden 
%%% ?berschrieben wird.
\fancypagestyle{plain}{%
\fancyhf{}
\lhead{\rightmark}
\rfoot{\thepage}
\renewcommand{\headrulewidth}{0.4pt} % unsichtbare Linie oben
\renewcommand{\footrulewidth}{0.4pt} % Linie mit 0.4pt unten
}

%\fancyhead{}
%\fancyfoot{}
%\lhead{\slshape \rightmark}
%\usepackage{lastpage}
%\usepackage{fancyhdr}
%\pagestyle{fancy}
%\renewcommand{\sectionmark}[1]{\markright{\thesection. #1}}
%\addtolength{\headheight}{3pt} %1
%\renewcommand{\sectionmark}[1]%
%             {\markright{\thesection\ #1}}
%\lhead[\fancyplain{}{\thepage}]%
%         {\fancyplain{}{\rightmark}}
%\rhead[\fancyplain{}{}]%
%             {\fancyplain{}{}}
%\chead[\fancyplain{}{}]%
%             {\fancyplain{}{}}
%\cfoot{} 													%Bei zweiseitigem Druck sollte 
%%\rfoot[\fancyplain{}{\rightmark}]	%das rausgenommen werden weil die Seitenzahlen
%%																		%sonst immer rechts angezeigt werden
%%             {\fancyplain{}{\thepage\ von \pageref{LastPage}}}

% ==========================================================
% Hurenkinder raus
% ==========================================================
\clubpenalty = 15000
\widowpenalty = 15000 \displaywidowpenalty = 15000

% ==========================================================
% Nach \\ und \newline nicht einr?cken
% ==========================================================
%\setlength{\parindent}{0em}

% ==========================================================
% Nummerierung bis subparagraph, Gliederung bis paragraph
% ==========================================================
\setcounter{secnumdepth}{5}
\setcounter{tocdepth}{4}

% ==========================================================
% Standardschriftart auf Helvetica/Arial (sonst Times New Roman)
% ==========================================================
\usepackage{helvet}
\renewcommand{\familydefault}{\sfdefault} % Neuladen der Schrift

% ==========================================================
% Zeilenabstand 1,5
% ==========================================================
\onehalfspacing

% ==========================================================
% Fu?noten richtig einr?cken, Zahl hochgestellt
% ==========================================================
\makeatletter
\newlength\footnoteindent
\newlength\footnotenumwidth
\newcommand*\footnotenumalign{r}
\newcommand*\footnoteformat{}

\renewcommand\footnoteformat{\textsuperscript} % Zahlen hochstellen

\setlength\footnoteindent{0.7em}%\parindent}
\settowidth\footnotenumwidth{999}%
\renewcommand\@makefntext[1]{%
	\@setpar{%
		\@@par
		\@tempdima\hsize
		\advance\@tempdima-\footnoteindent
		\parshape\@ne
		\footnoteindent\@tempdima
	}%
\parindent 1em
\par\noindent
\makebox[0pt][r]{% make everything lap into left margin
	\makebox[\footnoteindent][r]{%
		\makebox[\footnotenumwidth][\footnotenumalign]{%
			\footnoteformat{\@thefnmark}%
			}%
			\hspace*{\fill}%
		}%
	}%
	#1%
}
\makeatother

% ==========================================================
% neue Zeile nach paragraph und subparagraph
% ==========================================================
\makeatletter
\renewcommand\paragraph{\@startsection{paragraph}{4}{\z@}%
         {-3.25ex\@plus -1ex \@minus -.2ex}%
         {.5ex \@plus .2ex}%<--- Zeilenvorschub
         {\normalfont\normalsize\bfseries}}
\renewcommand\subparagraph{\@startsection{subparagraph}{4}{\z@}%
         {-3.25ex\@plus -1ex \@minus -.2ex}%
         {.5ex \@plus .2ex}%<--- Zeilenvorschub
         {\normalfont\normalsize\bfseries}}
\makeatother

% ==========================================================
% Bild- und Tabellenunterschriften links
% ==========================================================
\usepackage[nooneline]{caption2}
\makeatletter
%\captionstyle{left}
%\captionstyle{center}
\makeatother

% ==========================================================
% Index
% ==========================================================
\usepackage{makeidx}
\makeindex
% Der Index wird in TeXnicCenter per Ausgabe -> MakeIndex erstellt 
% (Dazu muss man sich im Hauptdokument befinden)
% Evtl. ist ein mehrfaches Erstellen des Dokuments n?tig

% ==========================================================
% Glossar
% ==========================================================
 \usepackage[style=super, header=none, border=none, number=none, cols=2,
 toc=true]{glossary}

 \makeglossary
 \renewcommand{\glossaryname}{Glossar} % Damit wird Glossary in Glossar umgetauft
 
% Externer Befehl. 
%(In TeXnicCenter kann dieser Befehl z.B. nach dr?cken
% von ALT +F7 im Bereich --MakeIndex als Kommando hineinkopiert werden. 
% nach Erstellung des Glossars muss dann unbedingt wieder der Standardwert "%bm" in dieses Feld eingetragen werden, da sonst die Erstellung des Indexes nicht funktioniert)

% makeindex -s DateinameOhneEndung.ist -t DateinameOhneEndung.glg -o DateinameOhneEndung.gls DateinameOhneEndung.glo

% ==========================================================
% PDF-Einstellungen (muss als letztes Paket geladen werden!)
% ==========================================================
\usepackage[
%   % Farben fuer die Links
   colorlinks=true,         % Links erhalten Farben statt Kaesten
   urlcolor=pdfurlcolor,    % \href{...}{...} external (URL)
   filecolor=pdffilecolor,  % \href{...} local file
   linkcolor=pdflinkcolor,  %\ref{...} and \pageref{...}
   citecolor=pdfcitecolor,  %Farbe von Zitaten
%   % Links
%   raiselinks=true,			 % calculate real height of the link
   breaklinks,              % Links berstehen Zeilenumbruch
   backref=page,            % Backlinks im Literaturverzeichnis (section, slide, page, none)
   pagebackref=true,        % Backlinks im Literaturverzeichnis mit Seitenangabe
%   hyperindex=true,         % backlinkex index
   linktocpage=true,        % Inhaltsverzeichnis verlinkt Seiten
%   hyperfootnotes=false,     % Keine Links auf Fussnoten
   % Bookmarks
   bookmarks=true,          % Erzeugung von Bookmarks fuer PDF-Viewer
   bookmarksopenlevel=1,    % Gliederungstiefe der Bookmarks
   bookmarksopen=true,      % Expandierte Untermenues in Bookmarks
%   bookmarksnumbered=true,  % Nummerierung der Bookmarks
%   bookmarkstype=toc,       % Art der Verzeichnisses
%   % PDF Informationen
   pdftitle={Latex-Vortrag},             % Titel
	 pdfsubject={Eine LaTeX-Einf�hrung},
   pdfauthor={Sebastian Stechl},            % Autor
]{hyperref}

% ==========================================================
% Linkfarben im PDF
% ==========================================================
% Farben fuer die Links im PDF
%\definecolor{pdfurlcolor}{rgb}{0,0,0.6}
%\definecolor{pdffilecolor}{rgb}{0.7,0,0}
%\definecolor{pdflinkcolor}{rgb}{0,0,0.6}
%\definecolor{pdfcitecolor}{rgb}{0,0,0.6}

% PDF-Linkfarben auf schwarz f?r den Druck:
 \definecolor{pdfurlcolor}{rgb}{0,0,0}
 \definecolor{pdffilecolor}{rgb}{0,0,0}
 \definecolor{pdflinkcolor}{rgb}{0,0,0}
 \definecolor{pdfcitecolor}{rgb}{0,0,0}


% ==========================================================
% Benennung von Literaturverzeichnis und Abstrakt
% ==========================================================
\addto\captionsngerman{
	\renewcommand{\abstractname}{Zusammenfassung}
	\renewcommand{\refname}{Literaturverzeichnis}
	\renewcommand{\lstlistingname}{Code}
	}

% ===========================================
% Toleranz beim Trennen erh?hen (default=200)
% ===========================================
\tolerance=1000
\newcommand{\bindestrich}{\discretionary{-}{}{-}}

% ==========================================================
% Selbstdefinierte Befehle
% ==========================================================

% Eurozeichen (als ? schreiben)
% \newcommand{\eur}{\euro\hspace{0.3em}}
\newcommand{�}{\euro} % Eurozeichen direkt eigeben
	
% ==========================================================
% ENDE DER VOREINSTELLUNGEN
% ==========================================================
